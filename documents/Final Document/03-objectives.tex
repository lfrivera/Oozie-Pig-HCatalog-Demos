El objetivo general del proyecto consiste en analizar, en términos de atributos de calidad, algunas de las herramientas que conforman el ecosistema de Hadoop. Particularmente, en este proyecto se analizará MapReduce, Apache Pig, Apache Hive, HCatalog, y Apache Oozie desde la perspectiva de los atributos de calidad de desempeño (\textit{performance}), reusabilidad (\textit{reusability}), extensibilidad (\textit{extensibility}), y mantenibilidad (\textit{maintainability}). A continación se describen los objetivos específicos del proyecto y los atributos de calidad asociados a cada uno de estos. \\

\begin{enumerate}

\item
{
Evidenciar la diferencia en los tiempos de ejecución de MapReduce, Apache Pig y Apache Hive. Atributo de calidad relacionado: \textit{performance}.
}

\item
{
Evidenciar la escalabilidad de Apache Pig. Atributo de calidad relacionado: \textit{scalability}.
}

\item
{
Evidenciar la extensibilidad de Apache Pig mediante HCatalog y su impacto en el desempeño de las ejecuciones. Atributos de calidad relacionados: \textit{performance, extensibility}.
}

\item
{
Evidenciar la reusabilidad y mantenibilidad de los workflows definidos con Apache Oozie. Atributos de calidad relacionados: \textit{reusability, maintainability}.
}

\end{enumerate}

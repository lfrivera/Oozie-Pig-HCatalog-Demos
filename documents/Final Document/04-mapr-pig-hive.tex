En esta sección se presenta la comparación de los tiempos de ejecución de MapReduce, Apache Pig, y Apache Hive para el cálculo de la temperatura máxima registrada por año. 

\subsection{Ejecución con MapReduce}

\subsubsection{Preparación}

Inicialmente, es necesario compilar el código fuente del programa \textit{MaxTemperature} en su versión Java. Para hacer esto, es necesario clonar el repositorio del libro \textit{Hadoop: The Definitive Guide}\footnote{Repositorio provisto por Tom White en https://github.com/tomwhite/hadoop-book}. Una vez hecho lo anterior, se debe proceder a compilar los archivos fuente necesarios mediante \textit{Maven}.Finalmente, la ruta del archivo .jar compilado deberá establecerse en una variable de entorno llamada \textit{HADOOP\_CLASSPATH}. 

\begin{lstlisting}[linewidth=\columnwidth,breaklines=true]
//Clonación del repositorio.
git clone &https://github.com/tomwhite/hadoop-book.git&

// Compilación del código fuente.
mvn package &-&DskipTests

// Definición del classpath de Hadoop.
export HADOOP_CLASSPATH&=&/home/sas6/Oozie&-&Pig&-&HCatalog&-&Demos/assets/&hadoop-&examples&.&jar
\end{lstlisting}


\subsubsection{Ejecución del programa MaxTemperature}


\begin{lstlisting}[linewidth=\columnwidth,breaklines=true]
hadoop MaxTemperature /user/&hive&/warehouse/weather_external/full_data&.&txt out_mr_300GB
\end{lstlisting}

\subsubsection{Seguimiento a la ejecución del programa}


\subsection{Running Pig}

\subsection{Running Hive}